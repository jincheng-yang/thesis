% meta.tex
% Copyright 2016 Zheng Xie <xie.zheng777@gmail.com>
% https://github.com/Tedxz/xjtuthesis-x
%
% This work may be distributed and/or modified under the
% conditions of the LaTeX Project Public License, either version 1.3
% of this license or (at your option) any later version.
% The latest version of this license is in
%   http://www.latex-project.org/lppl.txt
% and version 1.3 or later is part of all distributions of LaTeX
% version 2005/12/01 or later.
%
% This work has the LPPL maintenance status `maintained'.
%
% The Current Maintainer of this work is Zheng Xie.
%
% xjtuthesis-x is a Derived Work of xjtuthesis. The original maintainer of
% xjtuthesis is Weisi Dai (multiple1902 <multiple1902@gmail.com>),
% who published the project on https://code.google.com/p/xjtuthesis/ (no
% longer accessable). Currently, xjtuthesis is maintained by Aetf, and can
% be accessed on https://github.com/Aetf/xjtuthesis.
%
% xjtuthesis-x includes bug fixes, new features and a user guide.
% For detail, please refer to Readme.md.
%
% If you want to contribute to xjtuthesis-x or become the maintainer of
% xjtuthesis-x, please feel free to contact me.

% 标题,中文
\ctitle{Hodgkin-Huxley神经元晶格模型中Gamma突触输入诱导时空斑图现象的模拟}

% 作者,中文
\cauthor{张璐昊}

% 学科,中文,本科生不需要
% \csubject{}

% 导师姓名,中文
\csupervisor{康艳梅}

% 关键词,中文。用全角分号「;」分割
% 研究生的应首先从《汉语主题词表》中摘选
\ckeywords{Euler方程;无粘衰减}

% 提交日期,本科生不需要
% \cproddate{\the\year 年\the\month 月}

% 论文类型,中文,本科生不需要
% 从理论研究、应用基础、应用研究、研究报告、软件开发、设计报告、案例分析、调研报告、其它中选择
% \ctype{理论研究}

% 论文标题,英文
\etitle{xxxxxxxxxxxxx}

% 作者姓名,英文
\eauthor{Luhao Zhang}

% 导师姓名,英文
\esupervisor{Yanmei Kang}

% 关键词,英文。用半角分号和一个半角空格「; 」分割
\ekeywords{Euler Equation; Inviscid Damping}

% 学科门类,英文,本科生不需要
% 从Philosophy(哲学)、Economics(经济学)、Law(法学)、Education(教育学)、Arts(文学)、
%   Science(理学)、Engineering Science(工学)、Medicine(医学)、Management Science(管理学)中选择
% \ecate{Science}


% 摘要,中文。段间空行
\cabstract{
本文研究了具有指数密度分层的Couette流在二维半空间与管状空间内的线性稳定性问题. 在全空间内, 该问题的渐近稳定性与无粘衰减速率已经得到一个比较完整的分析. 相比于全空间而言, 研究更符合物理背景的半空间内和管状空间内方程的无粘衰减的文献则较少. 本文对于证明半空间与管状空间内Euler方程在Sobolev函数空间中的稳定性与衰减速率进行了初步尝试. 针对不同情况, 我们分别得到了在一定正则性的要求下的衰减速率与稳定性. 在一定条件下, 速度场在$L^2$意义下会有水平方向$t^{-\frac{1}{2}+\nu}$, 竖直方向有$t^{-\frac{3}{2}+\nu}$的衰减速率, 或者至少在非周期方向上的投影具有这样的衰减速率. 得到的这些衰减速率与全空间中的结论一致, 尽管它提出了一些不太合理的更高的正则性要求. 对于另一些情形, 有文献证明了流函数不存在衰减. 本文对这样的特征周期解则给出了显式的构造. 之后, 本文阐明了半空间和管状空间的解有着相似的结构. 我们还研究了该Hamilton系统的若干不变量, 这为将来的研究工作带来了一些启发. 然而, 这些不变量有无穷多个负方向, 所以无法用来证明方程的非线性稳定性.
}

% 摘要,英文。段间空行
\eabstract{
We study the linear stability of an exponentially stratified Couette flow in a half space and in a finite channel. A thorough analysis of the linear asymptotic stability and inviscid damping rate had been carried out for the whole space case. Comparing to the whole space, the study about the half space or the finite channel, which suits better the physics background, are relatively inadequate. This paper makes a naive attempt to show the linear stability and inviscid decay rate of the solutions to Euler equation in the half space and in the finite channel as functions in the Sobolev space. Decay rates of the solutions with certain regularity requirements or stability are obtained under different circumstances. Under certain assumptions, the velocity field has a decay rate in $L^2$ sense of $t^{-\frac{1}{2}+\nu}$ in the horizontal direction, and $t^{-\frac{3}{2}+\nu}$ in the verticle direction, or at least the projection perpendicular to the neutral spectrum has such decay rates. These rates are basically consistent with the decay rate in the whole space, although some not quite reasonable assumptions must been applied to reach the regularity requirements. For other cases, literature has shown nonexistance of decay for the stream function. We give explicit construction of such periodic eigensolutions. Afterwards, similar structure of the solutions in a half space and in a finite channel are compared. We also studied the Hamiltonian structure and invariants of the system, which brought some inspiration for the study. Unfortunately, all these invariants have infinitely many negative direction, hence they are unable to make any contribution for the proof of nonlinear stability.
}


% 致谢
\acknowledgement{
本文是在西安交通大学的李东升老师与佐治亚理工学院的林治武老师的共同指导下完成的, 受到美国国家科学基金会NSF (grantDMS-1411803)项目的资助. 林治武老师为论文的选题, 文献资料的选择提供了十分有益的指导, 并十分关照本人的工作进展, 使得论文得以顺利完成. 在佐治亚理工学院实习期间, 林治武老师组织的丰富的讨论班与讲座也给予了我很多启发. 同时也需要在此感谢学校拔尖办给予我此次出访机会, 感谢佐治亚理工学院的同僚们的帮助, 让我能够进行这样一项很有意义的科研. 另外, 本毕业设计虽然是在外校完成的, 李东升老师仍然给予了相当细致的关照与充分的支持, 本文的完成离不开他的帮助. 谨向林治武老师与李东升老师致以衷心的感谢! 
}